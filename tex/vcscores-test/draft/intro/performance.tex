\section*{Performance Notes}

The sources for this edition are performing parts that, on the one hand,
were used as practical tools for performance in a particular place, and, on the
other hand, represent traditions of performance that cannot be completely fixed
in place or time.
In other words, even within one institution, such as the Conceptionist convent
in Puebla from which come the parts for Salazar's \worktitle{Angélicos coros}
and Cáseda's \worktitle{Qué música divina}, these parts were used and reused
possibly over generations. 
In some cases, later performers made corrections, added barlines, sewed in new
lines of lyrics or even new music to replace certain strophes.
There is no single way that these pieces were performed throughout their terms
of service as part of the local repertoire.

Moreover, these pieces represent single instances of a repertoire that
circulated around the globe. 
José de Cáseda lived in Zaragoza and set a poem by a composer from his same
region, Vicente Sánchez; but his setting is only known from the surviving parts
in the Puebla convent.
The spelling in those parts reflects New Spanish, not Zaragozan pronunciation
(e.g., \mentioned{consonancias} is spelled \mentioned{consonansias} in the
Puebla parts, even though the final C would probably have been pronounced like an
English TH sound in Zaragoza).
The piece may have been rearranged or adapted for female ensemble from a lost
original with different scoring.
On some occasions, a particular sister may have fallen ill and her vocal line
may have been played instrumentally.

The starting point for considering modern performance of these pieces, then, is
that historic performers made these pieces their own and performed them in a way
that fit their local needs in terms of personnel, instrumentation, rehearsal
time, acoustic space, and other factors; and in a way that was intelligible and
meaningful to them and to their hearers.
In the case of pronunciation, for example, Spanish is the second most widely
spoken language in the world today, and nearly every city in the United States,
and many rural areas, include significant Spanish-speaking communities.
Ensembles should work with local native speakers whenever possible to shape
their pronunciation and understanding, so that they can perform these pieces in
a way that Spanish-speaking audience members will understand and recognize as a
part of their own cultural heritage.

In the case of instrumentation and voicing, ensembles should not be deterred by
the lack of Spanish \term{bajones}, \term{chirimías}, or cross-strung chromatic
harp; nor by vocal ranges outside their resources.
It would be entirely within the spirit of the performing traditions that these
sources represent, to substitute modern instruments like bassoon, oboe, guitar,
digital keyboard with a high-quality organ sample of 8$'$ flue stops, or just
piano; or to transpose the pieces to a more suitable register.

At the same time, performers are urged never to lose site of the religious,
social, and political contexts of these pieces in their early modern origins.
A piece intended for Eucharistic devotion need not only be performed in a
Roman Catholic liturgy; but it should not be presented with a sense of respect
for its devotional meanings.
Given that one in three humans identifies as Christian, these pieces may well
have personal religious meaning for many contemporary audience members.

By the same token, a piece like Juan Gutiérrez de Padilla's \worktitle{Al
establo más dichoso} bears the imprint of imperial Spain's racial hierarchy: it
is documented that the composer himself owned an Angolan slave, and the
representation of \quoted{Angolans} in the piece caricatures their bodies and
voices as deformed and deficient, even as it perhaps strives to present them in
a sympathetic light as offering devotion to Christ and joining with the angelic
chorus.
It would be ethically irresponsibly to perform such a piece merely as an exotic
curiosity, or worse, as though it were a twenty-first century celebration of
ethnic diversity. 
Indeed, performers, scholars, and community members ought to engage in serious
discussions about what performing a piece might mean in a contemporary context.
In the right setting, such as a community workshop with appropriate
opportunities for critique, response, and discussion, the piece might be used 
effectively to raise issues of tremendous contemporary relevance; but in the
wrong context the piece could actually perpetuate the negative racial
stereotypes that are built into it.

In the following sections, I provide a brief and somewhat speculative suggestion
for practices that would be close to seventeenth-century performances, followed
by some ideas for modern alternatives where these are needed.
Think of these like the ingredient substitutions in a cookbook: there is really
no way to substitute for \term{poblano} chiles (the kind cultivated in Puebla),
which provide both a rich vegetable flavor and a moderate amount of spicy heat;
but bell peppers capture some of the flavor and red pepper flakes create some of
the heat.
If you can get \term{poblanos} from a Hispanic grocery, by all means do so; but
no matter what ingredients you have, try to follow the recipe and make something
you and your family will enjoy.

\subsection*{Spanish Pronunciation}

\begin{epitome}
    Pronounce the text in the variety of modern Spanish that is most relevant to
    the performing ensemble's local community, consulting with native speakers.
\end{epitome}

Músicos en comunidades hispanohablantes deben usar la variedad del español que
se entiende por la mayoría de la audiencia.
No se procupe por problemas de \term{ceceo} o del orígen geográfico del
original, sino que el texto cantado no se cambia de lo escrito.

For ensembles with a majority of non-Spanish-speakers, consult with a native
speaker or Spanish teacher to develop a Spanish pronunciation that a local
audience of Spanish-speakers would understand. 
(Or recruit more Spanish-speaking members!)
In north America it makes the most sense to use a Latin American accent of
Spanish, with \term{ce-}, \term{ci-}, and \term{z} pronounced with an S sound.

The word \mentioned{villancico} is pronounced \mentioned{bee- yahn- SEE- ko} in
Latin America and \mentioned{bee- yahn- THEE- ko} (with a TH as in
\mentioned{thick}) in Spain.

For a full guide to pronunciation, please consult a good dictionary and a native
speaker.
The following differences between English and Spanish pronunciation should be
emphasized:

\begin{itemize}
    \item Vowels are pure Continental vowels like Italian, with no unwritten
        diphthongs or glides
    \item Ch is the same as English
    \item D, T, and P are not exploded
    \item L, M, and N are clearer (as in Italian)
    \item LL is like English Y
    \item Ge-, Gi- are an aspirated English H sound (like German \mentioned{Ich})
    \item J is a harsher aspirated English H sound (like German \mentioned{Bach})
    \item Gue sounds like "Gay" and Que sounds like "Kay"
        \item R is flipped or trilled 
        \begin{itemize}
            \item \mentioned{para} (single R between vowels) has a flipped r, more
                like the T sound in \quoted{water} or \quoted{leader}, and not like the
                British or American R sounds in those words
            \item \mentioned{ríos} (R at beginning of word) has a trilled R: use
                flipped R or even D sound, never an American R as in
                \mentioned{rivers}
            \item In \mentioned{guerrero}, the RR is rolled and the R in the last
                syllable is flipped
        \end{itemize}
    
    \item V is like English B
        \begin{itemize}
            \item \mentioned{vuestro} is like \mentioned{bwestro}, with a flipped r
        \end{itemize}

    \item D is usually soft like the TH in father
        \begin{itemize}
            \item \mentioned{de} is like \mentioned{they} 
            \item \mentioned{suspended} is pronounced \term{sus- pen- $'$deð}.
        \end{itemize}

    
    \item Ce-, Ci-, and Z are like S in Latin America and unvoiced TH in Spain
        \begin{itemize}
            \item \mentioned{voces} is like \term{bo- ses} in modern Latin America, 
                and \term{bo- thes} (TH as in \term{thick}) in modern Spain
            \item \mentioned{consonancias} is like \term{consonansias} in Latin America,
                \term{consonanthias} in Spain
        \end{itemize}
\end{itemize}



\subsection*{Voicing}

\begin{epitome}
    Use any combination of voices and instruments such that there is at least
    one singing voice per chorus to present the text.
    If possible, use soloists or a reduced ensemble for the first chorus in
    polychoral pieces, and for the coplas.
\end{epitome}

These pieces were originally performed by either an all-male chorus (as in a
cathedral or monastery ensemble, the Royal Chapel, or the school choir of
Montserrat) in which boys, adolescents, or falsettists sang the high voices; or
an all-female chorus in a monastic community.
The chorus of Puebla Cathedral in the 1650s may have had as many as forty
singers; other ensembles may have sung one-to-a-part.

The first chorus in polychoral pieces may have been sung by soloists or a
reduced ensemble of more highly skilled singers.
The coplas in many pieces were sung by soloists or smaller vocal group.

Tody the pieces could be performed by male, female, or mixed choruses, depending
on range and transposition (see below).
A chorus of more modest ability, such as a high school choir, could be paired
with more advanced soloists, such as college students or adult community
members.

\subsection*{Instrumentation}

\begin{epitome}
    At a minimum, use a contrapuntal instrument like keyboard for the continuo
    and a melodic instrument like the historic \gloss{bajón}{dulcian} or modern
    bassoon or cello for the instrumental bass lines.

    If more instruments are available, add other instruments to the continuo
    section like historic \term{bajón}, \term{vihuela}, Spanish harp, and organ;
    or modern alternatives like bassoon, cello, modern harp, guitar, digital
    organ, or piano.
    Double vocal parts freely with historic \gloss{bajoncillos}{small dulcians},
    \gloss{chirimías}{shawms}, \gloss{sacabuches}{sackbuts}, or possibly
    \gloss{vihuela de arco} or other bowed strings; or with modern alternatives
    like bassoons, oboes, trombones, or strings.
   
    There is no known documentation for the use of percussion instruments in
    this repertoire.
    The editor recommends that you resist any urges to exoticize these pieces by
    adding a flamenco or Afro-Cuban twist.

    Instrumental parts and continuo realizations are available from the editor
    upon request.  
\end{epitome}
    

\subsection*{Pitch Level}

\begin{epitome}
    Many of the pieces are scored at a very high pitch level, reflecting either
    a lost art of high singing, a lower historic pitch level, practices of
    transposition on sight.
    Sing the pieces at a pitch level or transposition that works for your
    ensemble.
    Transposed scores are available from the editor upon request.
\end{epitome}

One of the features of seventeeth-century Spanish choral music that is most
surprising to those more familiar with other repertoires is how high the vocal
ranges are.
Many of the Tiple (treble) parts have tessituras above \pitch{F}{5}; and none of
the pieces have texted bass parts, these parts being played instrumentally
instead.
Either Spanish ensembles performed these pieces at a lower pitch level than
notated (because of a lower general pitch, or through transposition), or Spain
cultivated a lost art of angelically high singing.

A full investigation of this question would incorporate research on
\term{chiavette}, corpus analysis of Spanish choral ranges in different
locations across the empire, knowledge gained from surviving wind instruments
and organs, historical data about body size and nutrition (relevant to
estimating how high boys in particular could sing), and contemporary scientific
research into vocal anatomy and pedagogy.
The Escolania of Montserrat, with its unbroken tradition of training boys'
voices, would be a logical starting place for that research.

Until that work is done, it seems most likely that (1) the general level of
pitch was probably significantly lower than A440, and the level was not the same
everywhere; (2) some ensembles may have transposed on sight (e.g., imagining a
change of clef to modulate down a fourth from \term{cantus mollis} to
\term{cantus durus}); (3) Spanish boys and women really did learn to sing
in extraordinarily high ranges by modern standards (since it is physically
possible for human beings to sing in these ranges with proper training).

Practically, if the range is a problem, the editor recommends that ensembles
sing these pieces as written but with a lower general pitch level; or that the
pieces be transposed if necessary to suit an ensemble.
The editor will make transposed scores and parts available upon request.


\subsection*{Rhythm and Meter}

\begin{epitome}
    The sign \meterC{} indicates a duple meter that should be felt \quoted{in
    two}.
    The sign \meterCThree{} indicates a ternary meter that should be felt
    \quoted{in one}.
    Often triple meter is syncopated or altered by hemiola (also called
    \term{sesquialtera}) to create patterns of accentuation that differ from the
    normal ternary groupings indicated by the barlines.

    In most cases, keep a tempo relationship of three minims in \meterCThree{}
    to one minim in \meterC{}.
    Perform \meterC{} at about $\minim = 60$ and \meterCThree{} at about
    $\semibrevedotted = 60$ ($\minim = 180$).
\end{epitome}


The original sources use two metrical signs, \meterC{} and \meterCZorig{}
(\meterCZ).
In both meters, there is one rhythmic impulse per metrical unit
(\term{tactus} in Latin, \term{compás} or measure in Spanish sources).

The original notation did not require barlines.
Most of the sources are performing parts with one voice per part, and no bar
lines showing the division of measures.
A few parts do have barlines written in, not always consistently; and the
surviving scores such as those by Miguel de Irízar do have barlines
(generally written with barlines between every two metrical measures).

This edition preserves the original rhythmic values to the extent possible
in modern notation, without any reduction---a minim in the original is a
minim in the edition).
The edition uses a standard \meterC{} meter symbol for duple meter and
\meterCThree{} for triple meter, and places barlines dividing each metrical
unit---one \term{compás} or \term{tactus} in the original is one numbered
measure in the edition.

The first sign, C meter, indicates a duple meter in which there is one semibreve
(whole note) per measure, which is divided into two minims (half notes).
The conductor or singer's hand falls and rises once per measure.
Musicians are encouraged to think of the duple-meter sections as equivalent
to \meter{2}{2} rather than \meter{4}{4}.

The second sign, \meterCZorig{} or \meterCZ, is a cursive shorthand for
C\meter{3}{2} or C3. 
This meter is rendered throughout this edition as C3.
This is a triple meter in which there is one perfect semibreve (dotted whole
note) per measure, which is divided into three minims (half notes).
The conductor or singer's hand falls and rises once per measure, but
unevenly (typically, down for the first minim and up for the other two).
Musicians are encouraged to think of the triple-meter sections as
essentially \quoted{in one}, equivalent to \meter{3}{2}.

In C3, Spanish composers often make heavy use of rhythmic alterations in
order to declaim vernacular texts in a nuanced way, or to evoke existing
song and dance styles.
In the absence of barlines, composers signalled deviations from the normal
ternary groups through filling in the round noteheads with ink, a technique
known as mensural coloration.
Such passages are notated with square brackets in the edition, and it is
important to note when the groupings created by the brackets are at odds
with the groupings created by the barlines.
Often this is used simply for an offbeat accent or syncopation.

In other cases, it creates a \term{hemiola} or \term{sesquialtera} pattern,
which sometimes alternates or contrasts against normal tenary motion.
In \term{sesquialtera}, two groups of three minims are replaced by three
groups of two minims; or two perfect semibreves are replaced by three
imperfect (colored) semibreves.


In most cases, a $3:1$ relationship of \emph{tempo} between \meterCThree{}
and \meterC{} makes the most musical sense, so that three minims in
\meterCZ{} together take the same amount of time as one minim in \meterC{}.
This means that a perfect semibreve (dotted whole note) in \meterCThree{}
would equal a minim (half note) in \meterC{}.
Most of the pieces work well at a tempo that matches the resting pulse 
(a heart rate of about sixty beats per minute) with the minim in \meterC{}
or the whole measure in \meterCThree{}.
In other words, perform \meterC{} at about $\minim = 60$ and \meterCThree{}
at about $\semibrevedotted = 60$ ($\minim = 180$).

The editor recommends avoiding exaggerated efforts either to stress every
downbeat or to avoid doing so.
Instead, pay close attention to the declamation of poetic text, which in
most cases should determine patterns of emphasis (keeping in mind that in a
few cases, like the mule-driver's song and the \term{Negrilla} in Juan
Gutiérrez de Padilla's \worktitle{Al establo más dichoso}, words seem to be
deliberately misaccentuated for purposes of characterization).

A final note about rhythm in villancicos: if there are African or American
indigenous rhythms in these pieces, no one has yet provided evidence to prove
it.
Similarly, some of the pieces do seem to refer to existing song or dance styles
(such as the \term{Nuevo Troyano} and \term{Papalotillo} in Padilla's
\worktitle{Al establo}), but there is not yet evidence to confirm these
references aside from the music in this edition.





