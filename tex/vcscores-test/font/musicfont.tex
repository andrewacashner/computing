\documentclass[12pt]{article}
\usepackage{ebgaramond-complete}
\usepackage[T1]{fontenc}
\usepackage{semantic-markup}
\usepackage[bigger]{musicography}
\usepackage[prime]{octave}

\renewcommand{\musNumFont}[1]{\liningnums{#1}}
\renewcommand{\fl}{\musFlat}
\renewcommand{\sh}{\musSharp}
\renewcommand{\na}{\musNatural} 

\begin{document}

At the end, the indication \emph{D.S. al Fine} points back to a \musSegno{} sign.
The tempo in C is $\musMinim = 60$ or in CZ, ($\musSemibreveDotted = 60$,
$\musMinim = 180$).
Is there a \musSemiminim, a \musEighth, or a \musSixteenth?

Let's switch to \meterC{} or \meterCThree{} or \meterCThreeTwo, but not
\meterCutC. \meterCZ{} is represented the same as \musMeter{3}{2}.

He wrote a B\fl, a C\sh, and a D\na. (not B$\flat$, C$\sharp$, or D$\natural$).

A semitone above middle C is \pitch{C}[\sh]{4}; almost bottom of the
keyboard is \pitch{B}[\fl]{0}, another note is \pitch{D}{5}.

\end{document}
