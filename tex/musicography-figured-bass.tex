\documentclass{article}
\usepackage{musicography}
\begin{document}

Seventeenth-century Spanish composers did not have a natural (\na) symbol, so
to raise the B\fl{} in \emph{cantus mollis} they would write B\sh{}.

In the same period, \meterC{} meant there were two primary minim pulses per
\emph{tactus} (\musMinim{} \musMinim{}), like modern \musMeter{2}{2}.
For ternary meter, \meterCThree{} had three minims per \emph{tactus}
(\musMinim{} \musMinim{} \musMinim{}), like modern \musMeter{3}{2}.

Vivaldi's \emph{Nulla in mundo pax sincera} features a recurring sequence of
progressions from \musFig{6 3} to \musFig{5 3} with an ascending stepwise bassline.

A common cadence in 18th-century Viennese music features a \musFig{6--5 4--3}
suspension and resolution over the dominant.

\end{document}
