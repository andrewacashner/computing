% Test of XeTeX or LuaTeX without system fonts

\documentclass[12pt]{article} 

\usepackage{libertine}
\usepackage{fontspec}
\usepackage{amsmath,amssymb,wasysym}

\usepackage[american]{babel}

\usepackage[margin=1in]{geometry}
\usepackage[doublespacing]{setspace}
\usepackage[none]{hyphenat}
\usepackage[document]{ragged2e}
\frenchspacing\sloppy
\setlength{\RaggedRightParindent}{0.5in}

\usepackage{csquotes}
\usepackage[notes,backend=biber,bibencoding=utf8]{biblatex-chicago}
\addbibresource{../../phd/biblio/phd.bib}

%*******************
% Special characters

% Accidentals
\newfontfamily{\musicfont}{Tex Gyre Pagella Math}
\newcommand{\textmusic}[1]{\bgroup\musicfont #1 \egroup}
\newcommand{\fl}{\textmusic{\char"266D}}
\newcommand{\na}{\textmusic{\char"266E}}
\newcommand{\sh}{\textmusic{\char"266F}}
% Music figures
\newcommand{\musfig}[2]{$\genfrac{}{}{0pt}{1}{\text{#1}}{\text{#2}}$} 
\newcommand{\xfl}{\hphantom{\fl}}
\newcommand{\xna}{\hphantom{\na}}
\newcommand{\xsh}{\hphantom{\sh}}

% Elision undertie
\newcommand{\undertie}{\raisebox{-1ex}{$\smallsmile$}}

% Academic dots
\newcommand{\Dots}{[\dots]}

% GREEK, HEBREW
\newcommand{\textgreek}{}
\newcommand{\texthebrew}{}

%*******************
\begin{document}

eß---em dash--en dash; manual accent esdr\'julo and unicode esrújulo.
\textgreek{a} \& \textgreek{w} signo\undertie a (\emph{la, mi, re})

% from ams 
coloration symbols: $\ulcorner$ and $\urcorner$.
%{} only when the space after a macro means a literal space
\fl6 \fl 6 C\sh, D\na\ plus E\fl{}!
%from waysym
Jupiter \jupiter\ Moon \leftmoon \rightmoon
Copyright \copyright 2014 Andrew A. Cashner, unicode © 2014.

In the Septuagint, the names of these two instruments are translated as the \textgreek{ψαλτήριον} (``psalterion'') and \textgreek{κιθάρα} (``kithara'').
These \Dots{} are dots.
For Christian theology, the most prominent Biblical locus for the Greek \textgreek{κινύρα} and Latin Vulgate \emph{cithara} is in the New Testament Revelation to John.
In the Vulgate, the instrument David plays for Saul in 1 Sam 16:16 is a \emph{cithara}; in Hebrew this was again 
\texthebrew{
כִּנּוֹר 
} (``kinnor''), rendered in the Septuagint with another frequently used translation, κινύρα (``kinura'').

%********************
\begin{quote}
%
\textgreek{καὶ ἡ φωνὴ ἣν ἤκουσα ὡς κιθαρῳδῶν κιθαριζόντων ἐν ταῖς κιθάραις αὐτῶν.}\\
%
\emph{et vocem quam audivi sicut citharoedorum citharizantium in citharis suis}\\
%
and I heard the voice of harpers harping with their harps
%
\end{quote}
%********************

``Counted among the cithara-players of God are all the saints, who, having crucified their flesh with its vices and sinful desires praise God with the psalter and cithara.''%
	%
	\footnote{
	\autocite[429]{Ribera:Apocalypse}: ``Cum citharista Dei omnes sancti sint, qui carnem suam crucifigentes cum vitijs \& concupiscentiis laudenteum in psaltero \& cithara'' (from a Google Books scan of a copy bearing the \emph{ex libris} marking ``from the library of the noviciate of the Company of Jesus in Madrid'').
	} 
	%

``Peregrino tono'' may mean ``wandering song,'' ``strange song (or tone),'' or the plainchant psalm tone the \emph{tonus peregrinus}.

\musfig{\fl6}{\xfl4}--\musfig{\xna5}{\na3}. musfig

Sharping the F would produce strange results--
\musfig{\fl6}{\sh3}--\musfig{4}{2}--\musfig{\xsh3}{\sh1} sonority.
in the Tiple and Altus m.~73, where the \fl6 resolves upward; and in m.~74, since the Tiple can only be F\na, a nearly unthinkable diminished octave between the same two voices. 
Both Fs occur in melodic patterns (G--F--G) where a singer might be tempted to sing the F as an F\sh{} according to \emph{musica ficta} performance practice, but an F\na{} makes more sense in both places.

\end{document}
