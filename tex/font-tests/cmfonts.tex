\documentclass[11pt]{article} 

\usepackage[T1]{fontenc}
\usepackage[utf8]{inputenc}
\usepackage[greek,english]{babel}
\usepackage[margin=1in]{geometry}
\usepackage[doublespacing]{setspace}
\usepackage[none]{hyphenat}
\usepackage[document]{ragged2e}
\frenchspacing\sloppy

\usepackage{csquotes}
\usepackage[notes,backend=biber,bibencoding=utf8]{biblatex-chicago}
\addbibresource{../../phd/biblio/phd.bib}
\newcommand{\fl}{$\flat$}
\newcommand{\sh}{$\sharp$}
\newcommand{\na}{$\natural$}
\newcommand{\stack}[2]{$#1 \atop #2$}

%\newfontfamily\greekfont[Script=Greek, Ligatures=TeX]{Linux Libertine}
%\newcommand{\textgreek}[1]{\bgroup\greekfont #1\egroup}
%\newcommand{\textgreek}{}

%\newfontfamily\hebrewfont[Script=Hebrew, Ligatures=TeX]{Linux Libertine}
%\newcommand{\texthebrew}[1]{\bgroup\luatextextdir TRT\hebrewfont #1\egroup}
%\usepackage{comment}
%\newcommand{\texthebrew}[1]{\begin{comment}#1\end{comment}}

%*******************
\begin{document}

eß

\section{%
Hebrew and Greek
}

In Gen. 4:21 this is one of the first instruments ever invented by ``Jubal, the ancestor of all those who play the lyre and pipe'' 
%(NRSV, directly translating the Hebrew: \texthebrew{ 
%כִּנּוֹר וְעוּגָ
%}).
In the Septuagint, the names of these two instruments are translated as the \textgreek{ψαλτήριον} (``psalterion'') and \textgreek{κιθάρα} (``kithara'').
St. Jerome rendered this passage in the Latin Vulgate with the words \emph{cithara et organo}, transliterating the Greek \textgreek{κιθάρα} and perhaps trying to recuperate the sense of a wind instrument in the other Hebrew word, which the Septuagint translators had turned into a stringed instrument, the ``psalterion.''

This example is one of several that demonstrate how the precise meaning of a musical term could be lost in the transfer between Hebrew, Greek, and Latin.
%In the Vulgate, the instrument David plays for Saul in 1 Sam 16:16 is a \emph{cithara}; in Hebrew this was again 
%\texthebrew{
%כִּנּוֹר 
%} (``kinnor''), rendered in the Septuagint with another frequently used translation, κινύρα (``kinura'').
The same transfer of terms happens in the descriptions of worship in the Davidic tabernacle (1 Chron. 15:28, 16:4--6): in Hebrew David's temple musicians play the ``kinnor''; in Greek, the ``kinura,'' and in the Vulgate they play \emph{citharae}.%
	%
\footnote{%
I am grateful to Hebrew-Bible scholar Drayton Benner for his help with the Hebrew. 
}
%

For Christian theology, the most prominent Biblical locus for the Greek \textgreek{κινύρα} and Latin Vulgate \emph{cithara} is in the New Testament Revelation to John.
In Rev. 14:2--4, John hears a chorus of 144,000 virgins singing ``a new song before the throne,'' and both the Greek and the Latin almost onomatopoetically echo their sound.
Among English translations only the 1611 King James version (shown below) tries to render this literally.

%********************
\begin{quote}
%
\textgreek{καὶ ἡ φωνὴ ἣν ἤκουσα ὡς κιθαρῳδῶν κιθαριζόντων ἐν ταῖς κιθάραις αὐτῶν.}\\
%
\emph{et vocem quam audivi sicut citharoedorum citharizantium in citharis suis}\\
%
and I heard the voice of harpers harping with their harps
%
\end{quote}
%********************

Many medieval exegetes commenting on the Latin Vulgate likely had no idea (or desire to know) what actual instrument the term \emph{cithara} referred to, and concerned themselves instead with analogical interpretations.
By the seventeenth century, the term had become a rich node of allegorical connections.
In a 1603 commentary on the Apocalypse (Revelation), the Jesuit Francisco Ribera draws on the venerable Bede to interpret the cithara played by the saints in Rev. 14 as symbolic of the saints' bodily mortification: 
``Counted among the cithara-players of God are all the saints, who, having crucified their flesh with its vices and sinful desires praise God with the psalter and cithara.''%
	%
	\footnote{
	\autocite[429]{Ribera:Apocalypse}: ``Cum citharista Dei omnes sancti sint, qui carnem suam crucifigentes cum vitijs \& concupiscentiis laudenteum in psaltero \& cithara'' (from a Google Books scan of a copy bearing the \emph{ex libris} marking ``from the library of the noviciate of the Company of Jesus in Madrid'').
This is probably the symbolism behind the use of this reading for the Mass of the Feast of Holy Innocents (Jan. 28).
	} 
	%

Another Jesuit exegete, the influential Cornelius à Lapide, takes the connection between cithara and crucifixion farther.%
	%
	\footnote{
Lapide lived from 1567--1637, but his commentaries were published mostly after his death.
His commentary on Samuel was published in Antwerp in 1641, and the Madrid Biblioteca Nacional preserves a copy, E-Mn: 7/14091(1).
His works were collected and reissued in multiple editions through the eighteenth and nineteenth centuries.
His importance here, similar to that of Fray Luis de Granada, is both as a contemporary commentator who may have directly influenced Biblical thinkers in seventeenth-century Spain, and as a compendium of received tradition.
Lapide brings together a wide range of patristic and medieval traditions and thus serves as a convenient guide to older sources that could have influenced seventeenth-century thought independently of Lapide's actual publications.
	}
	%
In his commentary on 1 Sam. 16, when David (in the Latin version) plays the cithara to drive the demons away from King Saul, Lapide gives an epitome of the whole patristic and medieval tradition regarding the power of music over the affects.
Surveying the exegetical tradition to its earliest and most obscure sources---Lapide cites Angelomus of Luxeuil, Prosper of Aquitaine, Eucherius of Lyon, and Ambrosius Ansbertus---Lapide summarizes a traditional interpretation of the cithara:
``Allegorically, the cithara represents the cross of Christ; for just as the strings of a cithara are stretched out, thus Christ was stretched out on the cross.''%
	%
	\footnote{
	\autocite[370]{Lapide:1Samuel}: ``Allegorice, cithara repræsentat crucem Christi; sicut enim chordæ in cithara distenduntur, sic Christus distensus fuit in cruce.''
	}
	%
He cites Augustine (\emph{Sermo 3 de tempore}) to say that ``the cithara represents the flesh of Christ'': speaking apparently of the Greek three-stringed lyre, Augustine makes this a metaphor for the unity of the three Persons of the Holy Trinity, incarnate in the body of Christ.%
	%
	\autocite[370]{Lapide:1Samuel}.
	%
Once again it is apparent how the ambiguous meaning of the instrument name gave theologians great flexibility in interpreting the instrument symbolically.

%*******************
\section{%
Accidentals and Stacked Numerals
}

In the coplas, Padilla matches the \emph{conceptismo} of the poem with similarly intricate musical symbols.
As we have discussed, ``peregrino tono'' may mean ``wandering song,'' ``strange song (or tone),'' or the plainchant psalm tone the \emph{tonus peregrinus}.
The sense of ``wandering'' may be represented in the harmonic structure, since the pattern of movement from triad to triad is somewhat unpredictable: the voices move quickly from triads on G (minor) to D but suddenly shift away in a 5--6 voice leading and wander through other triads (example~\ref{ex:Padilla-Voces-peregrino_tono}).
Through mm.~75--76 the goal of harmonic movement seems unclear---compared, say, to the first choir's opening statement in the introducción, mm.~1--14; or to the more ``sustained'' harmonies in m.~77 on ``sustenidos.''
The sense of wandering is increased by the small flourish in the Tenor in the second minim of m.~74 and the melodic shape of the Tiple in mm.~74--76, leaping upwards, then coming down, and then ascending again, fittingly, on ``subió.''

The passage also contains a ``strange tone'' in that the copyist of the Altus part twice writes in sharps over the Fs (in m.~73 and 74; see figure~\ref{fig:Padilla-Voces-AI-MS-copla}), but sharping the F would produce strange results---a \stack{\flat6}{\sharp3} sonority in the Tiple and Altus m.~73, where the \fl{}6 resolves upward; and in m.~74, since the Tiple can only be F\na{}, a nearly unthinkable diminished octave between the same two voices. 
Both Fs occur in melodic patterns (G--F--G) where a singer might be tempted to sing the F as an F\sh{} according to \emph{musica ficta} performance practice, but an F\na{} makes more sense in both places.
So unless Padilla truly wanted to create a strange song with intentional dissonance, these sharps are most likely ``cautionary accidentals'' whose purpose (since Spanish notation had no natural symbol) is actually to warn the singer \emph{not} to sharp these Fs.%
	%
	\footnote{%
	\Autocites{Harran:Cautionary1}{Harran:Cautionary2}{Berger:Ficta}. 
Either way, the Fs may be thought of as ``wrong'' or ``false.''
The only way to tell the singer not to make the note ``false'' through ficta is to write a kind of false note: the copyist notates what the singer does \emph{not} sing. 
	}

\end{document}
