\documentclass{article}

\usepackage{libertine}
\usepackage[T1]{fontenc}
\usepackage[libertine]{newtxmath}

% Octave subscripts
\usepackage{subscript} % Needed for \textsubscript command
\newcommand{\octave}[1]{%
	\textsubscript{#1}%
}

% Accidentals
\newcommand{\fl}{$\flat$}
\newcommand{\sh}{$\sharp$}
\newcommand{\na}{$\natural$}

% Figured bass or voice leading, stacked numerals with accidentals
\usepackage{harmony}
\newcommand{\musfig}[2]{%
	\Takt{#1}{#2}%
}
% Blank accidental, needed to balance alignment when using figured bass with accidental in one voice only
\newcommand{\bl}{\hphantom{$\sharp$}}

\newcommand{\term}[1]{%
	\emph{#1}%
}
\usepackage{csquotes}
\newcommand{\socalled}[1]{%
	\enquote{#1}%
}

\begin{document}

According to \term{una nota super la} rule of \term{musica ficta}, a singer in colonial Mexico would in most cases perform the written pattern D--E--D as D--E\fl--D. 
Since Spanish notation at the time lacked a natural (\na) sign, a sharp sign could be added to the E as a \socalled{cautionary accidental}, so a singer reading D--E\sh--D would actually sing E\na.

The Alto's B\fl\octave{4} forms an imperfect consonance with the G\octave{3} in the Bass.
When the Bass moves down to F\sh{} on the downbeat, it creates a \musfig{\fl 6}{\sh 3} sonority (in the first mode).

Another voice-leading pattern is \musfig{\fl 6}{\bl 3}--\musfig{5}{3}.

\end{document}


