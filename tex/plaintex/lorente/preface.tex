% ANDRES LORENTE, EL PORQUE DE LA MUSICA (ALCALA DE HENARES, 1672)
% PREFACE
% Andrew Cashner, 2014-09-18
% Plain TeX

%************************************************************************

\hoffset=0.5in
\hsize=5.5in

\font\Hugefont=cmr10 at 48pt
\font\Largefont=cmr10 at 36pt
\font\largefont=cmr10 at 24pt
\font\bigfont=cmr10 at 16pt
\font\mainfont=cmr10 at 12pt
\mainfont
\def\Huge{\Hugefont \baselineskip=52pt}
\def\Large{\Largefont \baselineskip=40pt}
\def\large{\largefont \baselineskip=28pt}
\def\big{\bigfont \baselineskip=20pt}


\def\spreadtitleline #1 {\hbox to \hsize\bgroup#1\egroup \titlelineskip}
\def\centertitleline #1 {\line{\hss #1\hss}\titlelineskip}
\def\titlelineskip {\par\vskip 1ex}

%****************************************

\spreadtitleline{\Huge A MARIA SAN-}
\spreadtitleline{\Large TISSIMA, NVESTRA}
\spreadtitleline{\large ABOGADA, Y SE\~NORA, CONCE-}
{\big
\centertitleline{bida sin mancha de Pecado Original, en el}
\centertitleline{Primer Instante de}
\centertitleline{su Ser.}
}


\bye 

%*******************
% Compile with luatex --fmt luatex-plain
\def\scripture #1 {\bgroup\it #1\egroup\ }
\vskip 2em

Es la criatura racional, perfecta obra de las manos de Dios, en quien trasladó su semejanza; 
para que en sus lineas leyese la atencion judiciosa el acierto de su Soberana Sabiduria, 
epilogando en su animada Fabrica, quanto deve el ser a su Omnipotente brazo, 
porque admirase todo lo criado en una Imagen sola, cifradas sus perfecciones, 
haziendo en grata ostentación, lucido alarde de las semejanzas que copia; 
de Dios en el alma; de los Ángeles en el entendimiento; del Sol en el corazón; de la Luna en el cerebro; de los demás Planetas, en otras facultades; 
de los Elementos, en los humores; de los animales, en el sentir; y de las plantas en el crecer:
por cuya breve suma la llamaron los Antiguos, Microcosmos, o mundo abreviado; 
en cuya hermosa compostura brillan los rasgos de lo humano, alentados de diuino, superior espíritu, 
que imperioso preside al Consejo de Gobierno desta acorde, racional republica, 
en quien oficioso su natural movimiento, alterna con equidad su debido, individual ejercicio, 
sin que uno a otro miembro usurpe sus juridiciones, ni envidie su recto proceder.

Criatura racional fue Maria Santissima Señora Nuestra, en quien depositó Dios no solo su imagen, y semejanza, como hechura suya, 
sino juntamente todo el complemento de sus virtudes, gracias, y perrogativas (que no la hacen repugnancia)
dandose las liberalísimo Dios, para escogerla por Madre suya,
cuyo real privilegio gozó tan soberana, que calificó de nuevo la suma grandeza, y infinita sabiduria de la omnipotencia de Dios, 
en quien como en cristalino, inmaculado espejo reberveran los rayos del poder del Padre, de la Sabiduria del Hijo, y del amor del Santo Espiritu, 
a quien Seraphycas Esquadras venera reverentes, y adoran cortesanas, y de quien el Celeste estrellado pavimento mendiga resplandores para comunicarlo, 
prodigo en benignos amorosos influjos, por sus numerosos brillantes Astros; 
cuyas lenguas de fuego explican en luminosos caracteres lo augusto de su diadema:
\scripture{Et in capite eius corona stellarum duodecim}, a cuyo magestuoso imperio tributa el Sol un oriente de luzes, 
que en el bastidor del Supremo Olympo borda costosísima librea recamada de Celestiales Zaphyros: \scripture{Amicta Sole}.
Cuya luciente inundacion de rayos, postra en reverente obsequio la Luna a sus plantas, para que sirviendola de Sagrados Coturnos, sean sus pisadas seguro Norte de nuestro destino: 
\scripture{Et Luna sub pedibus eius}: que así vió a esta Gloriosísima Señora (como quieren todos los Expositores Sagrados) el remontado Espíritu del Coronista Joan.

