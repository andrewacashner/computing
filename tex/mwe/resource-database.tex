
Just a suggestion: Use your operating system as your database, and access it by storing file paths in LaTeX macros.

For example, use a directory structure like this:

    $HOME/
    $HOME/image
    $HOME/audio
    $HOME/video

In the `image` folder you have files like this:

    Bernini-Ecstasy_St_Teresa.png % or .pdf or .jpg
    Michelangelo-Last_Judgment.png
    Picasso-Guernica.png

In the `web` directory you would have text files containing URLs:

    `TeX-Stackexchange.tex` 
    
The contents of that file would be just this:

    `http://www.tex.stackexchange.com/`

Use macros like this:

    \newcommand{\homepath}{/home/you/}

    \newcommand{\imagepath}{\homepath/image}
    \newcommand{\audiopath}{\homepath/audio}
    \newcommand{\videopath}{\homepath/video}
    \newcommand{\webpath}{\homepath/web}

    \newcommand{\includeimage}[1]{%
        \includegraphics[width=\textwidth,height=0.3\textheight,keepaspectratio]{\imagepath/#1}%
    }
    \newcommand{\includeaudio}[1]{% however you include audio
    }
    \newcommand{\includevideo}[1]{% however you include video
    }
    \newcommand{\includeweblink}[1]{%
       \edef\newlink{\input{#1}}
       \href{\newlink}%
     }

Then in your document you can write:

    Michelangelo's \emph{Last Judgment} is a very big painting 
    (see figure~\ref{fig:Michelangelo-Last_Judgment}).
    
    \begin{figure}
    \caption{Michelangelo, \emph{The Last Judgment}}
    \label{fig:Michelangelo-Last_Judgment}
    \includeimage{Michelangelo-Last_Judgment} % macro automatically finds the path
    \end{figure} 

    For more information see \includeweblink{the TeX Stackexchange site}{TeX-Stackexchange}.

You could go farther and wrap the whole figure environment shown above in a macro:

    % #1 image file name, #2 caption
    \newcommand{\includeimagefloat}[2]{%
       \begin{figure}
       \caption{#2}
       \label{fig:#1}
       \includeimage{#1}
       \end{figure}%
    }

