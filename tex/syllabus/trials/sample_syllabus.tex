\documentclass{syllabus}

\usepackage{gfsdidot}
\usepackage[T1]{fontenc}
\usepackage[utf8]{inputenc}
\usepackage{exegesis}

\begin{document}

%****************************************
\setCourseTitle			{Music History I: Beginnings to ca.~1680}
\setCourseShortTitle	{Music History I}
\setCourseNumber		{MUHI~321}
\setCourseRoom			{MUS 123}
\setCourseTime			{MW 9:00--11:15 a.m.}
\setCourseTerm			{Fall 2015}
\setCourseInstitution	{University of Southern California, Thornton School of Music}

\printCourseInfo

%****************************************
\section*{Instructors}

\begin{PersonID}
\Role	{Professor}
\Name	{Andrew A. Cashner, PhD}
\Phone	{(111) 111-1111}
\Office	{MUS 304}
\Email  {cashner@usc.edu}
\end{PersonID}

\begin{PersonID}
\Role	{Teaching Assistant}
\Name	{Neda Kandimirova}
\Phone	{(222) 222-2222}
\Office	{MUS 304}
\Email	{neda.kandimirova@gmail.com}
\end{PersonID}

\printPersonList

%****************************************
\section*{Policies}

\subsection*{Grading}

This table lists the minimum percentage points needed to earn each grade (for example, 90--92 is A$-$ and 93--96 is A).

\setGradeRange{A}{90}{93}{97}	
\setGradeRange{B}{80}{83}{87}
\setGradeRange{C}{70}{73}{77}
\setGradeRange{D}{}{60}{}
\setGradeRange{F}{}{0}{}

\printGradeScale

%****************************************

\section*{Schedule of Classes and Assignments}

\section*{Unit I: Early Baroque, 1590--1680}
\subsection*{Opera}

%**************************************
\Week{Italy}
\Class[M]{Private opera}

\setAsstHeader{Reading}	{Reading:}
\setAsstHeader{Music}	{Music listening and score study:}

\begin{Asst}{Reading}
\Task[Primary]Strunk xx}
\Task[\Secondary]Heller 3}
\end{Asst}

\begin{Asst}{Music}
\Task{Monteverdi, \ti{L'Orfeo}}
\Task{Landi, \ti{Il Sant'Alessio}}
\end{Asst}

\begin{Asst}{Video}
\Task{Monteverdi, \ti{L'Orfeo} film}
\end{Asst}

%*******************
\Class[W]{Public opera}

\begin{Reading}
\Task[\Primary]{Strunk xx}
\Task[\Secondary]Heller 6}
\end{Reading}

\begin{Music}
\Task{Cavalli, \ti{Il Giasone}}
\Task{Cesti, \ti{La Griselda}}
\end{Music}


\end{document}
