\newcommand{\DisabilityPolicy}{%
\subsection*{Statement for Students with Disabilities}
Any student requesting academic accommodations based on a disability is required to register with Disability Services and Programs (DSP) each semester. 
A letter of verification for approved accommodations can be obtained from DSP. 
Please be sure the letter is delivered to me (or to TA) as early in the semester as possible. 
DSP is located in STU 301 and is open 8:30 a.m.--5:00 p.m., Monday through Friday. 
Website for DSP (\url{http://dsp.usc.edu/}) and contact information: (213) 740-0776 (phone), (213) 740-6948 (TDD only), (213) 740-8216 (fax) \url{ability@usc.edu}.%
}

\newcommand{\AcademicIntegrityPolicy}{%
\subsection*{Statement on Academic Integrity}
USC seeks to maintain an optimal learning environment. 
General principles of academic honesty include the concept of respect for the intellectual property of others, the expectation that individual work will be submitted unless otherwise allowed by an instructor, and the obligations both to protect one’s own academic work from misuse by others as well as to avoid using another’s work as one’s own. 
All students are expected to understand and abide by these principles. 
\ti{SCampus} (\url{http://scampus.usc.edu/}), the Student Guidebook, contains the University Student Conduct Code (see University Governance, Section 11.00), while the recommended sanctions are located in Appendix A.%
}

\newcommand{\EmergencyPolicy}{%
\subsection*{Emergency Preparedness; Course Continuity in a Crisis}
In case of a declared emergency if travel to campus is not feasible, USC executive leadership will announce an electronic way for instructors to teach students in their residence halls or homes using a combination of Blackboard, teleconferencing, and other technologies. 
See the university’s site on Campus Safety and Emergency Preparedness (\url{http://preparedness.usc.edu/}).%
}

\newcommand{\USCpolicies}{%
	\DisabilityPolicy
	\AcademicIntegrityPolicy
	\EmergencyPolicy%
}
