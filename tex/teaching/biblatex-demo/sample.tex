\documentclass{article}

\usepackage[notes]{biblatex-chicago}
\addbibresource{sample.bib}

\title{Melville}
\author{John Q. Student}

\begin{document}
\maketitle

I read Melville's \emph{Moby Dick}.\Autocite{Melville:MobyDick}
It uses a lot of symbolism.\Autocite{Smith:Symbolism}
A scholar talks about a specific symbol in one part.%
\Autocite[143--144]{Smith:Symbolism}
% With the comment character at the end of the sentence I can put the citation
% on the next line.

Here are two more specific references.%
\Autocites
[150]{Smith:Symbolism}
[74]{Gunderson:MelvilleWomen}
% That's just an example of tidily formatted source code.
Britten wrote a whole opera about it.%
\Autocite[s. v. \emph{Britten, Benjamin}]{GroveMusic}
% Instead of a page number I used the Latin abbreviation *sub verbum*
% to indicate which article to look up.

\printbibliography[title={Bibliography}]

\end{document}
