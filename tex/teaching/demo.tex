\documentclass{article}

\usepackage{lipsum} % this is how you include a package
% this one just generates dummy text in pseudo-Latin.

\usepackage{booktabs} % better table commands and layout

\title{Demonstration}
\author{Andrew Cashner}

\begin{document}
\maketitle

\tableofcontents
\listoftables

\section{Introduction}

This is a sample document created with the \LaTeX{} document-processing system.
It is good to start each sentence starts on a new line for more convenient editing and tracking changes.%
  \footnote{This is optional but recommended, just like putting footnotes on a separate line.}

A blank line starts a new paragraph.
There are various commands to mark up different types of text for \emph{emphasis}, \textbf{bold}, and ``quotations.''

\lipsum

\subsection{Environments}

To create a table, use an environment.
Table~\ref{environments} is an example, and this sentence is an example of a cross-reference.

\lipsum

\begin{table}

\begin{tabular}{ll} % two left-aligned columns
\toprule
Name & Description\\ 
\midrule
table & Creates a ``float'' for your table, positioned automatically\\
tabular & Align material vertically\\
quote & For a block quotation\\
verse & For poetry\\
\bottomrule
\end{tabular}

\caption{Environments}
\label{environments} % this establishes the cross-reference

\end{table}

\end{document}

