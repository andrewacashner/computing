\documentclass{usc-muhl-article}
%\setmainfont{Palatino} % use any OpenType or TrueType font installed on your system

\addbibresource{mybib.bib} % no spaces in file name, .bib extension

\title{Insert Title Here} % Insert information
\author{Insert Name Here}
% \date{Insert date other than today's date if needed} % Default is \today (current date)

\begin{document}
\maketitle

% This is a comment that won't be printed.
% Below is just dummy text for demonstration. 
Insert text here.
It's best to use a new line for each sentence.

Leave a blank line between paragraphs.%
  \footnote{Do footnotes like this.} 
Or do footnotes like this.%
  \begin{Footnote}
  Footnote text, with a citation. See \autocite{Robertson:PaleFace}.
  \end{Footnote}
Do bibliographic citations like this, using the key from your .bib file.%
  \autocite{Britten:Aspen} 

\section{Special Formatting}

Mark up technical terms like \term{this}.
Mark up quotations like ``this'' or---better---like \quoted{this}.
Write musical accidentals like B\fl{}, F\na{}, or C\sh{}.
Write meters or figured bass like this: \musfig{3}{4} meter, a \musfig{6--5}{4--3} cadential sequence.

\subsection{Foreign Characters}

Enter accent marks and foreign characters directly like \term{esdrújulo} or \worktitle{Sechs kleine Kavierstücke}; or you can write them with commands like esdr\'ujulo and Klavierst\"ucke.

\printbibliography

\end{document}