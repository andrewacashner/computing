\documentclass{article}

\begin{document}

\title{Representations of Dogs in Mongolian Harp Music}
\author{Ahmed A. Researcher}
\maketitle

\tableofcontents

\section{Introduction}

This is the introduction.
Every LaTeX document begins with a command that tells the program to use a particular set of predefined commands for a certain type of document.
In our case that's the article class, which is designed for academic journal articles.

After that, the body text of your document should be enclosed inside a `document' environment.
Put \verb|\begin{document}| at the beginning and \verb|\end{document}| at the end. 
The `verb' command I just used allows me to include commands as regular text instead of actually using them as commands. 
Don't worry about it for now.

After that, you just type your document.
You have lots of freedom in how you type it.
For example, it doesn't matter if you have extra   spaces   between   words,
  or if you go to a new line. The extra
    spaces will be ignored.
You can use the percent character to tell the program to ignore everything that follows % such as all these words
on the same line.
% Advice
Use these for comments that won't be printed in the document.

It's a good practice, though, to begin a new sentence on a new line.
That makes it easier to edit your document, and helps in thinking about writing structure.

Skip a line to start a new paragraph.
The program will automatically format the paragraph spacing, indentation, and so on.
After this I'll start a new section, and the program will format the heading according to pre-programmed default settings.

\section{Commands}

LaTeX is really a programming language based on macros.
Macros, also known as commands, tell the program to substitute something else in place of the macro.
When you type \LaTeX, the program prints its own name with a special format.
You can print today's date with the macro \today.

Macros can also have arguments: this is text that will be processed by the command, and is normally enclosed in curly brackets.
You've already seen this with the sectioning command above.
Another example: to put text in boldface font, you can use \textbf{word}. 
Whatever is inside the curly brackets after the command will be set in bold.

Another kind of command is an environment, such as the ``document'' environment we're using now.
Environments are used to group a whole section of text that needs special treatment, like a numbered list, which we'll cover below.

\subsection{Formatting Commands}

Macros are used to control formatting so that you can separate meaning from presentation.
You can emphasize something with \emph{this command}, which is pre-programmed to be set in italic font.
You can put something in quotation marks ``like this.''

You should also know the commands for different types of dashes:
type one dash for a hyphen as in free-spirited; 
type two dashes for an ``en dash'' as in a range of dates like 1987--1995;
type three dashes for an ``em dash'' to mark an interruption in thought---like this.


There are other font commands, like \textbf{bold}, \textsc{small caps}, and \MakeUppercase{upper case}.
You can define your own commands, and redefine existing commands, however you like.
It's better to define ``semantic commands'' to use instead of these, so you don't have to think directly about formatting as you write.
We'll do that later.

\section{Footnotes}

To write footnotes, we use a built-in command.\footnote{This is a footnote.}
It can be easier to maintain your document if you put footnotes on a separate line, and indent the footnotes to separate them from the rest.
The only trick with this is that you have to make sure no space is added between the text and the footnote number.
So just put a comment character (percent character) at the end of the line.%
  \footnote{See how I did that?}
That's easier to keep track of.

\section{Lists}

There are two kinds of lists: un-numbered bullet lists, created using an ``itemize'' environment; and numbered lists, made with the ``enumerate'' environment.
You could list three numbered items like this:
\begin{enumerate}
\item Thing one
\item Thing two
\item Thing three
\end{enumerate}

You could list three bullet-list items like this:
\begin{itemize}
\item Thing one
\item Thing two
\item Thing three
\end{itemize}

\end{document}

\section{Custom Commands}

It is simple to define your own commands.
The uses for this are endless, but the most common is to allow for ``semantic'' markup;
in other words, create commands to hide specific formatting so you can adjust it later.
The \verb|\emph{}| command is a good example.
The direct-formatting command would be \verb|\textit{}|; but with \verb|\emph| you don't think ``italics,'' you think ``emphasis.''
You could reprogram the command later to put all emphasized words in bold or something else.



\end{document}


