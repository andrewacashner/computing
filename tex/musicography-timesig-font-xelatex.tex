\documentclass{article}
\usepackage{fontspec}
\setmainfont{SourceSansPro-Light}[
UprightFont = SourceSansPro-Light,
ItalicFont = SourceSansPro-LightIt,
BoldFont = SourceSansPro-Regular
]
\newfontfamily\xlightfont{SourceSansPro-ExtraLight}[
UprightFont = SourceSansPro-ExtraLight,
ItalicFont = SourceSansPro-ExtraLightIt,
BoldFont = SourceSansPro-Light
]
%\renewcommand{\familydefault}{\sfdefault}
\usepackage{musicography}
\newfontfamily{\TimeSigFont}{Emmentaler-11}
\NewDocumentCommand{\TimeSig}{ m m }{%
    \musStack[\TimeSigFont]{#1 #2}%
}

\begin{document}
In regards to rhythm, the standard notation renderer has been enhanced at the
quantiser level.  
Here, instead of writing two \TimeSig{4}{4} measures (see Figure 3.12,
version 3.63), the algorithm writes eight \TimeSig{1}{4} measures (see
Figures 3.13a and 3.13b, version 3.85. 
This allows two things: 1) complex patterns are conveniently delineated by bar
lines, and thus easier to decipher; TEST: \musFlat{}, \musSharp{}, and
\musNatural{}, \musThirtySecond{}, \meterCThreeTwo{}, $8\times\TimeSig{1}{4}$
instead of $2\times\TimeSig{4}{4}$.
\end{document}
