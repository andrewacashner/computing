\documentclass{ms}
\usepackage{lwarp}
\usepackage[authordate]{biblatex-chicago}
\addbibresource{master.bib}

\title{My Article}
\author{Andrew Cashner}
\begin{document}
\maketitle
\section*{Abstract}

This is an article with sections, including different kinds of floats and
whatnot with cross references and bibliography.

\section{Introduction}

This is the introduction. It will not remind you of the preface to
\wtitle{Moby-Dick}. 
Or perhaps it will. 
\quoted{Call me a document} \autocite[1]{Kircher:Musurgia}.
It's up to you.%
\begin{Footnote}
    Follow the hermeneutic circle all the way to the end.
\end{Footnote}

What you'll see when you look at figure \ref{fig:squares} is a lot of squares.
They're all nested inside each other and get infinitely smaller.
I made it in LaTeX using a loop. 
You'll find it is different from what I will show you in section \ref{sec:next}.

\begin{figure}
    \label{fig:squares}
    \includegraphics[height=0.5\textheight]{img/squares.jpg}
    \caption{Infinite squares}
\end{figure}

\section{Next part \label{sec:next}}

What's different about these squares from the next thing I'll talk about is
that next thing isn't squares at all. 
It's notated music (example \ref{ex:crotchets}).
It shows \term{crotchets}, which is the British term for quarter notes, I think. 
Anyway, Benjamin Britten in \wtitle{Billy Budd} would like you to make sure
they keep \quoted{gently flowing}.
They don't have anything to do with \term{crocheting}, which is a handicraft
not frequently featured in British opera.%
\footnote{But why not?}

\begin{example}
    \label{ex:crotchets}
    \includegraphics[height=2\baselineskip]{img/crotchets.png}
    \caption{``Gently flowing crotchets''}
\end{example}

\printbibliography
\end{document}
