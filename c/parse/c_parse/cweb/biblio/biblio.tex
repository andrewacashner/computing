\input cwebmac
% 2015-05-03 -- biblio.w by Andrew A. Cashner

\N{1}{1}Introduction.
This is \.{biblio}, a program to build a bibliographic database from input with
minimal markup.
The goal is to allow a simple way to add to a database in \.{biblatex} format.

Normally a \.{biblatex} entry looks like this:

\.{\\fi

\M{2}Book{Kircher:Musurgia,
author   = {Kircher, Athanasius},
title    = {Musurgia universalis},
location = {Rome},
year     = 1650,
}}

In our system the user specifies in advance which fields will be included for
each entry, like so:

%\verbatim\PB{${\#}\\{au},\\{ti},\\{loc},\\{yr}\MOD\\{Kircher},\\{Athanasius}%
%\MOD\\{Musurgia}\\{universalis}\MOD\\{Rome}\MOD\T{1650}$}endverbatim

Each entry is separated by a blank line, just like paragraphs in \TeX.
A line beginning with \.{\#} specifies the contents of the fields for every
subsequent entry.
\TeX-style comment lines starting with \.{\%} are ignored.

In this way authors can add many entries in a similar format at once, adding
another \.{\#} instruction line if a new setup of fields is needed.
The user may specify a default set of fields by writing a field-instruction
line
that begins \.{\#DEFAULT}, like so:

\verbatim #DEFAULT au, ti, loc, pub, yr\PB{\\{endverbatim}\\{Then}\\{after}%
\\{changing}\\{the}\\{field}\\{instruction}\\{with}\|a\ $.$ $\{\$ $\#$ $\}$ %
\&{line} $,$ \\{the}\\{user}\\{may}\\{restore}\\{the}\&{default} \\{by}%
\\{writing}\|a \&{line} \\{with} \\{only}\ $.$ $\{\$ $\#\$ $\#$ $\}$ $.$ %
\\{The}\\{user}\\{may}\\{use}\\{the}\\{exact}\\{field}\\{names} (\ $.$ $\{%
\\{author}\}$ $,\$ $.$ $\{\\{title}\},{\Xand}\\{so}\\{on}$ ) $,$ ${\Xor}\|a{}$%
\&{default} \\{set}\\{of}\\{abbreviations} \&{for} \\{the} \\{fields} (\ $.$ $%
\{\\{au}\}$ $,\$ $.$ $\{\\{ti}\}$ ) $.$ \\{The}\\{user}\\{may}\\{also}%
\\{specify}\\{custom}\\{abbreviations} \&{using} \\{an}\${}.\{\{\#}\.{ABBREV}\}%
\\{instruction},{}$ \\{as}\\{follows} :\ $\\{verbatim}\#\.{ABBREV}\\{kw}{}$: %
\\{keyword}\\{od}: \\{origdate}}endverbatim

Each abbreviation is followed by a colon and then the full field name, and each
abbreviation is separated by a space.

% TODO give full example in separate file

\Y\B\4\D$\.{MAX\_FILENAME}$ \5
\T{50}\par
\fi

\M{3}Dummy program

\Y\B\8\#\&{include} \.{<stdio.h>}\6
\8\#\&{include} \.{<stdlib.h>}\7
\&{int} \\{main}(\&{int} \\{argc}${},\39{}$\&{char} ${}{*}\\{argv}[\,]){}$\1\1%
\2\2\6
${}\{{}$\1\6
\&{char} \\{infilename}[\.{MAX\_FILENAME}];\6
\&{char} \\{outfilename}[\.{MAX\_FILENAME}];\6
\&{FILE} \\{infile};\6
\&{FILE} \\{outfile};\7
\&{if} ${}(\\{argc}\I\T{3}){}$\5
${}\{{}$\1\6
${}\\{fprintf}(\\{stderr},\39\.{"Incorrect\ number\ of}\)\.{\ arguments.%
\\n"});{}$\6
\\{exit}(\.{EXIT\_FAILURE});\6
\4${}\}{}$\2\6
${}\\{printf}(\.{"Read\ file\ \%s\ and\ ad}\)\.{d\ to\ bib\ file\ \%s.\\n"},\39%
\\{argv}[\T{1}],\39\\{argv}[\T{2}]);{}$\6
\&{return} (\T{0});\6
\4${}\}\{}$\2\6
\\{bye}\par
\fi

\inx
\fin
\con
