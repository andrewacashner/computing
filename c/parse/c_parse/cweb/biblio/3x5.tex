\input cwebmac
% 2015-05-03 -- biblio.w by Andrew A. Cashner

\def\beginlisting{%
	\par\vskip\baselineskip%
	\begingroup%
	\tt\obeylines\obeyspaces%
	\catcode`\{12\catcode`\}12%
}
\def\endlisting{%
	\endgroup\par%
	\vskip\baselineskip%
}


\N{1}{1}Introduction.
This is \.{3x5}, a program to build a bibliographic database from input with
minimal markup.
The goal is to allow a quick way to add to a database in \.{biblatex} format,
as simple as writing down the info on a $3 \times 5$ notecard.

Normally a \.{biblatex} entry looks like this:

\beginlisting
@Book{Kircher:Musurgia,
\  author   = {Kircher, Athanasius},
\  title    = {Musurgia universalis},
\  location = {Rome},
\  year     = 1650,
}
\endlisting

In our system the user specifies in advance which fields will be included for
each entry, like so:

\beginlisting
\# book: au, ti, loc, yr
Kircher, Athanasius
Musurgia universalis
Rome
1650
\endlisting

Each entry is separated by a blank line, just like paragraphs in \TeX.
A line beginning with \.{\#} specifies the contents of the fields for every
subsequent entry.
\TeX-style comment lines starting with \.{\%} are ignored.

In this way authors can add many entries in a similar format at once, adding
another \.{\#} instruction line if a new setup of fields is needed.
The user may specify a default set of fields by writing a field-instruction
line
that begins \.{\#DEFAULT}, like so:

\beginlisting
\#DEFAULT book: au, ti, loc, pub, yr
\endlisting

Then after changing the field instruction with a \.{\#} line, the user may
restore the default by writing a line with only the entry type, omitting the
fields:

\beginlisting
\# book
\endlisting

The user may use the exact field names (\.{author}, \.{title}, and so on), or a
default set of abbreviations for the fields (\.{au}, \.{ti}).
The user may also specify custom abbreviations using an \.{\#ABBREV}
instruction,
as follows:

\beginlisting
\#ABBREV kw:keyword od:origdate
\endlisting

Each abbreviation is followed by a colon and then the full field name, and each
abbreviation is separated by a space.

% TODO give full example in separate file

\fi

\M{2}Macro definitions.

\Y\B\4\D$\.{MAX\_FILENAME}$ \5
\T{50}\par
\B\4\D$\.{BUFFER\_SIZE}$ \5
\T{1056}\par
\B\4\D$\.{MAX\_LINE}$ \5
\T{320}\par
\fi

\M{3}Dummy program

\Y\B\8\#\&{include} \.{<stdio.h>}\6
\8\#\&{include} \.{<stdlib.h>}\6
\8\#\&{include} \.{<string.h>}\7
\&{int} \\{main}(\&{int} \\{argc}${},\39{}$\&{char} ${}{*}\\{argv}[\,]){}$\1\1%
\2\2\6
${}\{{}$\1\6
\X4:Main variables\X\6
\X5:Check arguments from command line\X\6
\X6:Open files for reading and writing\X\6
\X8:Read one entry\X\6
\X9:Close files\X\6
\&{return} (\T{0});\6
\4${}\}{}$\2\par
\fi

\N{1}{4}Process input.

\Y\B\4\X4:Main variables\X${}\E{}$\6
\&{char} \\{infilename}[\.{MAX\_FILENAME}];\6
\&{char} \\{outfilename}[\.{MAX\_FILENAME}];\6
\&{FILE} ${}{*}\\{infile};{}$\6
\&{FILE} ${}{*}\\{outfile}{}$;\par
\A7.
\U3.\fi

\M{5}User specifies the name of an input file and a destination .bib file.

\Y\B\4\X5:Check arguments from command line\X${}\E{}$\6
\&{if} ${}(\\{argc}\I\T{3}){}$\5
${}\{{}$\1\6
${}\\{fprintf}(\\{stderr},\39\.{"Incorrect\ number\ of}\)\.{\ arguments.%
\\n"});{}$\6
\\{exit}(\.{EXIT\_FAILURE});\6
\4${}\}{}$\2\6
${}\\{printf}(\.{"Read\ file\ \%s\ and\ ad}\)\.{d\ to\ bib\ file\ \%s.\\n"},\39%
\\{argv}[\T{1}],\39\\{argv}[\T{2}]){}$;\par
\U3.\fi

\M{6}Open the input and output files.

\Y\B\4\X6:Open files for reading and writing\X${}\E{}$\6
$\\{strcpy}(\\{infilename},\39\\{argv}[\T{1}]);{}$\6
${}\\{strcpy}(\\{outfilename},\39\\{argv}[\T{2}]){}$;\C{ .bib must be included
explicitly }\6
${}\\{infile}\K\\{fopen}(\\{infilename},\39\.{"r"});{}$\6
\&{if} ${}(\\{infile}\E\NULL){}$\5
${}\{{}$\1\6
${}\\{fprintf}(\\{stderr},\39\.{"Could\ not\ open\ file}\)\.{\ \%s\ for\
reading.\\n"},\39\\{infilename});{}$\6
\\{exit}(\.{EXIT\_FAILURE});\6
\4${}\}{}$\2\6
${}\\{outfile}\K\\{fopen}(\\{outfilename},\39\.{"wr"});{}$\6
\&{if} ${}(\\{outfile}\E\NULL){}$\5
${}\{{}$\1\6
${}\\{fprintf}(\\{stderr},\39\.{"Could\ not\ open\ file}\)\.{\ \%s\ for\
writing.\\n"},\39\\{outfilename});{}$\6
\\{exit}(\.{EXIT\_FAILURE});\6
\4${}\}{}$\2\par
\U3.\fi

\N{1}{7}We read one bibliographic entry into a buffer.

\Y\B\4\X4:Main variables\X${}\mathrel+\E{}$\6
\&{char} \\{bibentry\_buffer}[\.{BUFFER\_SIZE}]; \&{char} \6
\&{line} [\.{MAX\_LINE}]\1\5
;\2\7
\&{int} \\{lines\_read};\6
\&{typedef} \&{enum} ${}\{{}$\1\6
${}\.{FALSE},\39\.{TRUE}{}$\2\6
${}\}{}$ \&{boolean};\6
\&{boolean} \\{entry\_end};\par
\fi

\M{8}Read one entry.

\Y\B\4\X8:Read one entry\X${}\E{}$\6
$\\{bibentry\_buffer}[\T{0}]\K\.{'\\0'};$ \&{while} ${}(\\{entry\_end}\E%
\.{FALSE})$ $\{$ \\{fgets} ( \&{line} $,$ \&{sizeof} ( \&{line} ) $,$ %
\\{infile} )  ; \&{if} ( \&{line} $\E$ $\NULL$ $\V$ \&{line} [\T{0}] $\E$ \.{'%
\\n'} ) \6
${}\{{}$\1\6
${}\\{entry\_end}\K\.{TRUE};{}$\6
\4${}\}{}$\2\6
\&{else} $\{$ \\{strcat} $(\\{bibentry\_buffer},\39$ \&{line} )  ;\6
${}\PP\\{lines\_read};$ $\}$ $\}$ $\\{printf}(\.{"\%d\ lines\ read\\n"},\39%
\\{lines\_read});{}$\6
${}\\{printf}(\.{"\%s\\n"},\39\\{bibentry\_buffer});{}$\6
${}\\{fputs}(\\{bibentry\_buffer},\39\\{outfile}){}$;\par
\U3.\fi

\M{9}Close files.

\Y\B\4\X9:Close files\X${}\E{}$\6
\\{fclose}(\\{infile});\6
\\{fclose}(\\{outfile});\par

\U3.\fi


\inx
\fin
\con
